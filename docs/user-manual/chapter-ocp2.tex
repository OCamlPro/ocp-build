

The OCP2 language is the new language used by ocp-build for package
descriptions.

\section{Statements and Expressions}

In the OCP2 language, there are two kinds of constructs:
\begin{itemize}
\item \emph{Statements} are constructs that can modify the current
  environment. For example, the statement \verb+x=1;+ will define a
  variable \verb-x- with value \verb-1- in the current environment;
\item \emph{Expressions} are constructs that simply compute a value,
  in the current environment, but without the ability to modify it.
\end{itemize}

\section{Values}

In the OCP2 language, values can take the following types:
\begin{itemize}
\item \emph{Integers} are standard OCaml integers;
\item \emph{Booleans} are either \verb+true+ or \verb+false+;
\item \emph{Strings} are standard OCaml strings;
\item \emph{Lists} are standard OCaml lists;
\item \emph{Tuples} are standard OCaml tuples;
\item \emph{Objects} are maps from field names to values, such as
  \verb+{ x=1; y="foo" }+.
\end{itemize}

Such values are the simplest expressions available in the
language.

Values are not mutable. For example, the following program prints
\verb+{ y = 1 }+:
\begin{verbatim}
x.y = 1;
z = x;
z.y = 2;
print(x);
\end{verbatim}

\section{Blocks of statements}

In the OCP2 language, there are two kinds of blocks of statements:
\begin{itemize}
\item Blocks between braces, such as \verb+{ x=1; y=f(x); }+, do not
  create a new environment. Instead, they are executed in the current
  environment, and return a new environment with all modifications
  applied;
\item Blocks between the \verb+begin+ and \verb+end+ keywords do not
  modify the current environment. Instead, they are executed in a copy
  of the current environment, that is discarded at the end of the
  block.
\end{itemize}

For example, in the following example, the program prints \verb+2,1+:
\begin{verbatim}
x = 1; y = 1;
{ x = 2; }
begin y=2; end
print(x,",",y);
\end{verbatim}

\section{Functions and Lexical Scope}

In the OCP2 language, functions are created as closures: variables
in the body of a function are interpreted within the environment
at their creation site, not at their execution site.

For example, in the following example, the program prints \verb+1+:
\begin{verbatim}
x = 1;
function f(){ return x; }
x = 2;
print(f());
\end{verbatim}

The value returned by a function is specified by a {\tt return} statement.

Functions are first order values, i.e. they can be defined by expressions,
and manipulated as any other value:
\begin{verbatim}
function incr(x){ return x+1; }
apply = function(f, arg){ return f(arg); };
print(apply(incr, 3));
\end{verbatim}



\section{Control Flow}

The following control-flow constructs are available:

\subsection{The {\tt if} statement}

The \verb+if+ statement allows the program to check a boolean
condition.  Here is a basic example:

\begin{verbatim}
if( x = 1 ){
   print("x equals one");
} else {
   print("x does not equal one");
}
\end{verbatim}

The \verb+else+ part is optional.

\subsection{The {\tt for...in} statement}

The \verb+for+ statement allows the program to iterate on a range of values.

Here are some examples:
\begin{verbatim}
list = [ 1;2;3;4 ];
for v in list { print(v); }
for v in (1,4) { print(v); }

list = [ 1;3;5;7 ];
for v in list { print(v); }
for v in (1,2,7) { print(v); }
\end{verbatim}

Note that the variable of the \verb+for+ will escape the last
execution of the body.

\subsection{The {\tt try...catch} statement}

The \verb+try+ statement allows the program to catch an exception:

\begin{verbatim}
try {  print(x); } (* x not defined *)
catch("unknown-variable",arg){
  print("Unknown variable: ", arg);
}

try {  x = {};  print(x.y); (* x.y not defined *) }
catch("hello",arg){ print("never here"); }
catch("unknown-field",arg){ print("Unknown field: ", arg); }
\end{verbatim}

An exception is always a tuple containing a string identifier and an
argument, so each {\tt catch} will only catch exceptions with the same
string identifier, and bind the exception argument to the provided
variable name in the body (and that variable will escape the handler).

The {\tt raise} primitive can raise a user-defined exception in user code:
\begin{verbatim}
raise("my-exception", "Message or any other value");
\end{verbatim}

\section{Using OCP2 to describe packages}

\ocpbuild{} will search {\tt build.ocp2} files in your
directories. For backward compatibility, it will also search for files
with the {\tt .ocp} extension (and parse them with its former
language). If both a {\tt build.ocp2} and {\tt build.ocp} files are
present, \ocpbuild{} will only execute one of them: it will choose the
{\tt build.ocp2} file by default, unless a {\tt .ocp-build} file is
present with the line {\tt maxversion=1}.

By convention, options for projects should be stored in an object
called {\tt ocaml}.

A typical package description will thus look like:

\begin{verbatim}
begin
  ocaml.files = [
    "foo.ml";
    "bar.ml", { inline = 10 };
  ];
  ocaml.requires = [ "ocplib-compat"; "ocplib-lang" ];
  new_package("my-lib", "library", ocaml);
end

begin
   ocaml.files = [ "main.ml" ];
   ocaml.requires = [ "my-lib" ];
   new_package( "my-program", "program", ocaml);
end
\end{verbatim}
