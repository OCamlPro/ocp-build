

The OCP2 language is the new language used by ocp-build for package
descriptions.

\section{Files and Evaluation}

There are two kinds of OCP2 files:
\begin{description}
\item[Description Files:] Description Files are files describing a
  project to be compiled by \ocpbuild{}. Each directory of a project
  can only contain one Description File, always called {\sf
    build.ocp2}.
\item[Module Files:] Module Files are files defining set of functions
  (i.e. modules) to be used in Description Files. Module Files cannot
  be called {\sf build.ocp2}.
\end{description}

The order of evaluation is the following:
\begin{itemize}
\item Module Files are evaluated first, in a non-specified order
\item Description Files are evaluated second, starting at the root,
  and descending in each subtree. A Description File in a
  sub-directory is always evaluated after a Description File in an
  ancestor directory, but the execution order is not specified between
  two Description Files in unrelated directories.
\end{itemize}

Module Files are expected to define modules using the {\sf provides}
function. A file {\sf path/namespace/module.ocp2} is expected to define
the module {\sf namespace:Module}. For example:

\begin{lstlisting}[language=ocp2]
function ListMaker(){
    List = {};
    try { List.length = value_length; } catch("unknown-variable", x){}
    try { List.mem = List_mem; } catch("unknown-variable", x){}
    return List;
}
provides("ocp-build:List", "1.0", ListMaker);
\end{lstlisting}

Such modules can then be used from Description Files:

\begin{lstlisting}[language=ocp2]
Sys = module("ocp-build:Sys", "1.0");
include_file = "build.ocpinc";
if( Sys.file_exists(include_file) ){ include(include_file); }
\end{lstlisting}

Note that the function {\tt ListMaker} in the example above will only
be executed once when the module {\tt ocp-build:List} will be
used. Using functions this way helps solving dependencies between
modules.

\section{Statements and Expressions}

In the OCP2 language, there are two kinds of constructs:
\begin{itemize}
\item \emph{Statements} are constructs that can modify the current
  environment. For example, the statement \verb+x=1;+ will define a
  variable \verb-x- with value \verb-1- in the current environment;
\item \emph{Expressions} are constructs that simply compute a value,
  in the current environment, but without the ability to modify it.
\end{itemize}

\section{Values}

In the OCP2 language, values can take the following types:
\begin{itemize}
\item \emph{Integers} are standard OCaml integers;
\item \emph{Booleans} are either \verb+true+ or \verb+false+;
\item \emph{Strings} are standard OCaml strings;
\item \emph{Lists} are standard OCaml lists;
\item \emph{Tuples} are standard OCaml tuples;
\item \emph{Objects} are maps from field names to values, such as
  \verb+{ x=1; y="foo" }+.
\end{itemize}

Such values are the simplest expressions available in the
language.

Values are not mutable. For example, the following program prints
\verb+{ y = 1 }+:
\begin{verbatim}
x.y = 1;
z = x;
z.y = 2;
print(x);
\end{verbatim}

\section{Blocks of statements}

In the OCP2 language, there are two kinds of blocks of statements:
\begin{itemize}
\item Blocks between braces, such as \verb+{ x=1; y=f(x); }+, do not
  create a new environment. Instead, they are executed in the current
  environment, and return a new environment with all modifications
  applied;
\item Blocks between the \verb+begin+ and \verb+end+ keywords do not
  modify the current environment. Instead, they are executed in a copy
  of the current environment, that is discarded at the end of the
  block.
\end{itemize}

For example, in the following example, the program prints \verb+2,1+:
\begin{verbatim}
x = 1; y = 1;
{ x = 2; }
begin y=2; end
print(x,",",y);
\end{verbatim}

\section{Functions and Lexical Scope}

In the OCP2 language, functions are created as closures: variables
in the body of a function are interpreted within the environment
at their creation site, not at their execution site.

For example, in the following example, the program prints \verb+1+:
\begin{verbatim}
x = 1;
function f(){ return x; }
x = 2;
print(f());
\end{verbatim}

The value returned by a function is specified by a {\tt return} statement.

Functions are first order values, i.e. they can be defined by expressions,
and manipulated as any other value:
\begin{verbatim}
function incr(x){ return x+1; }
apply = function(f, arg){ return f(arg); };
print(apply(incr, 3));
\end{verbatim}



\section{Control Flow}

The following control-flow constructs are available:

\subsection{The {\tt if} statement}

The \verb+if+ statement allows the program to check a boolean
condition.  Here is a basic example:

\begin{verbatim}
if( x = 1 ){
   print("x equals one");
} else {
   print("x does not equal one");
}
\end{verbatim}

The \verb+else+ part is optional.

\subsection{The {\tt for...in} statement}

The \verb+for+ statement allows the program to iterate on a range of values.

Here are some examples:
\begin{verbatim}
list = [ 1;2;3;4 ];
for v in list { print(v); }
for v in (1,4) { print(v); }

list = [ 1;3;5;7 ];
for v in list { print(v); }
for v in (1,2,7) { print(v); }
\end{verbatim}

Note that the variable of the \verb+for+ will escape the last
execution of the body.

\subsection{The {\tt try...catch} statement}

The \verb+try+ statement allows the program to catch an exception:

\begin{verbatim}
try {  print(x); } (* x not defined *)
catch("unknown-variable",arg){
  print("Unknown variable: ", arg);
}

try {  x = {};  print(x.y); (* x.y not defined *) }
catch("hello",arg){ print("never here"); }
catch("unknown-field",arg){ print("Unknown field: ", arg); }
\end{verbatim}

An exception is always a tuple containing a string identifier and an
argument, so each {\tt catch} will only catch exceptions with the same
string identifier, and bind the exception argument to the provided
variable name in the body (and that variable will escape the handler).

The {\tt raise} primitive can raise a user-defined exception in user code:
\begin{verbatim}
raise("my-exception", "Message or any other value");
\end{verbatim}

\section{Basic Primitives}

Here is a list of available primitives:
\begin{description}
\item[module(name,version):] retrieves a user-defined module with name
  and minimal version. The name must contain a namespace (i.e. {\tt
    "ocp-build:List"}). Can raise {\sf "not-found", "bad-version",
    "infinite-dependency"};
\item[provides(name,version):] defines a new module with name and
  version;
\item[print(...):] prints its arguments on the terminal
\item[message(string):] prints the string on the terminal
\item[exit(code):] exits from \ocpbuild{}
\item[raise(name,arg):] raises an exception {\sf name} (a string) with
  some argument.
\item[value\_type(v):] returns a string containing the type of {\sf
  v}: one of {\sf "string", "list", "tuple", "object", "bool", "int",
  "function", "prim"}
\item[version(s):] transforms a string into a version. Versions are
  strings, compared using version comparison à la Debian.
\item[store\_set(name, v):] associates {\sf v} with string {\sf name}
  in a shared store;
\item[store\_get(name):] retrieves the value associated with string
  {\sf name} in a shared store. Raises {\sf "not-found"} if it is not
  found;
\end{description}

\section{Basic Modules}

The {\sf List} module:
\begin{lstlisting}[language=ocp2]
  List.length(list)
  List.mem(ele, list)
  List.tail(list)
  List.map(function,list)
  List.map(list,function)
  List.flatten(list of lists)
\end{lstlisting}

The {\sf Sys} module:
\begin{lstlisting}[language=ocp2]
  Sys.file_exists(filename)
  Sys.readdir(dirname)
  Sys.readdir(dirname, regexp)
\end{lstlisting}

The {\sf Store} module:
\begin{lstlisting}[language=ocp2]
  Store.get(name)
  Store.set(name)
  Store.counter()  returns { get(); reset() }
  Store.counter(name)
\end{lstlisting}

The {\sf String} module:
\begin{lstlisting}[language=ocp2]
  String.mem(ele, list)
  String.length(s)
  String.concat(list of strings)
  String.concat(sep, list of strings)
  String.concat(list of strings, sep)
  String.split(string, char)
  String.split_simplify(string, char)
  String.subst_suffix(string, old_suffix,new_suffix)
  String.subst_suffix(list of strings, old_suffix,new_suffix)
  String.subst(string, assocs)
\end{lstlisting}

The {\sf Features} module: used to interact with the {\sf --with} and
{\sf --without} arguments on the command line.
\begin{lstlisting}[language=ocp2]
  Features.get(name_string, default_bool)
  Features.enable(name_string, enable_bool)
  Features.with(name_string, value_bool)
  Features.without(name_string)
\end{lstlisting}


\section{Using OCP2 to describe packages}

\ocpbuild{} will search {\tt build.ocp2} files in your
directories. For backward compatibility, it will also search for files
with the {\tt .ocp} extension (and parse them with its former
language). If both a {\tt build.ocp2} and {\tt build.ocp} files are
present, \ocpbuild{} will only execute one of them: it will choose the
{\tt build.ocp2} file by default, unless a {\tt .ocp-build} file is
present with the line {\tt maxversion=1}.

By convention, options for projects should be stored in an object
called {\tt ocaml}.

A typical package description will thus look like:

\begin{verbatim}
begin
  ocaml.files = [
    "foo.ml";
    "bar.ml", { inline = 10 };
  ];
  ocaml.requires = [ "ocplib-compat"; "ocplib-lang" ];
  new_package("my-lib", "library", ocaml);
end

begin
   ocaml.files = [ "main.ml" ];
   ocaml.requires = [ "my-lib" ];
   new_package( "my-program", "program", ocaml);
end
\end{verbatim}

\section{The OCaml Plugin}

The OCaml plugin defines two global variables: the module {\sf OCaml} and
the record {\sf ocaml}.
